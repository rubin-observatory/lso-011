\documentclass[OPS,obsolete,toc]{lsstdoc}

% lsstdoc documentation: https://lsst-texmf.lsst.io/lsstdoc.html

% Package imports go here.

% Local commands go here.

% To add a short-form title:
% \title[Short title]{Title}
\title{OBSOLETE see RDO-11 . Release Scenarios for LSST Data}

% Optional subtitle
% \setDocSubtitle{A subtitle}

\author{%
William O'Mullane, Phil Marshall, Leanne Guy
}

\setDocRef{LSO-011}
\setDocStatus{Obsolete}

\date{\today}

% Optional: name of the document's curator
% \setDocCurator{The Curator of this Document}

\setDocAbstract{%
SEE RDO-11 this doc is obsolete.
A first go at describing some release scenarios for LSST
}

% Change history defined here.
% Order: oldest first.
% Fields: VERSION, DATE, DESCRIPTION, OWNER NAME.
% See LPM-51 for version number policy.
\setDocChangeRecord{%
  \addtohist{1}{YYYY-MM-DD}{Unreleased.}{William O'Mullane, Phil Marchall, Leanne Guy }
}

\begin{document}

% Create the title page.
% Table of contents is added automatically with the "toc" class option.

\mkshorttitle
%switch to \maketitle if you wan the title page and toc


% ADD CONTENT HERE ... a file per section can be good for editing
\input{body}

\appendix
% Include all the relevant bib files.
% https://lsst-texmf.lsst.io/lsstdoc.html#bibliographies
\section{References} \label{sec:bib}
\bibliography{lsst,lsst-dm,refs_ads,refs,books}

%Make sure lsst-texmf/bin/generateAcronyms.py is in your path
\section{Acronyms used in this document}\label{sec:acronyms}
\addtocounter{table}{-1}
\begin{longtable}{p{0.145\textwidth}p{0.8\textwidth}}\hline
\textbf{Acronym} & \textbf{Description}  \\\hline

ComCam & The commissioning camera is a single-raft, 9-CCD camera that will be installed in LSST during commissioning, before the final camera is ready. \\\hline
DAC & Data Access Center \\\hline
DMTN & DM Technical Note \\\hline
DPDD & Data Product Definition Document \\\hline
DR & Data Release \\\hline
DR1 & Data Release 1 \\\hline
DR10 & Data Release 10 \\\hline
DR11 & Data Release 11 \\\hline
DR2 & Data Release 2 \\\hline
FITS & Flexible Image Transport System \\\hline
LDM & LSST Data Management (document handle) \\\hline
LSE & LSST Systems Engineering (Document Handle) \\\hline
LSO & LSST Science Operations (document handle) \\\hline
LSST & Legacy Survey of Space and Time (formerly Large Synoptic Survey Telescope) \\\hline
MPC & Minor Planet Center \\\hline
MREFC & Major Research Equipment and Facility Construction \\\hline
NCSA & National Center for Supercomputing Applications \\\hline
OPS & OPerationS \\\hline
RDO & Rubin Directors Office \\\hline
TBD & To Be Defined (Determined) \\\hline
US & United States \\\hline
ZTF & Zwicky Transient Facility \\\hline
\end{longtable}

\end{document}
